\documentclass{article}
\usepackage[english]{babel}
\usepackage[a4paper,top=2.5cm,bottom=2.5cm,left=1.55cm,right=1.55cm,marginparwidth=1.0cm]{geometry}
\usepackage{xcolor}
\usepackage{lastpage}
\usepackage{datetime}

% -- Useful packages
\usepackage{amsmath}
\usepackage{bm}
\usepackage{graphicx}
\usepackage[none]{hyphenat}\hyphenpenalty=10000\tolerance=500
\usepackage[utf8]{inputenc}
\usepackage{fancyhdr}\pagestyle{fancy}\fancyhf{}\lhead{{PolyModels Hub Confidential}}\rhead{Milling - E2E Model Development}\cfoot{Page \thepage \,\,of \pageref{LastPage}}

\usepackage[colorlinks=true, allcolors=blue]{hyperref}\setlength{\parindent}{0em}
\usepackage{upgreek}
\usepackage{mathrsfs}
\usepackage{float}
\usepackage[]{hyperref}
\hypersetup{linkcolor=black}
% Define a new page style for the first page
\fancypagestyle{firstpage}{
  \fancyhf{} % clear all header and footer fields
  \renewcommand{\headrulewidth}{0pt} % remove header line
  \renewcommand{\footrulewidth}{0pt} % remove footer line
}

% Uncomment the following line for dark mode in pdf
% \usepackage{xcolor} \pagecolor[rgb]{0,0,0} \color[rgb]{1,1,1}

%------------------------------------------------------------------------------------------------------------------------------------------------------------------------------------------------------------
% Main document starts here:
%------------------------------------------------------------------------------------------------------------------------------------------------------------------------------------------------------------

\begin{document}
\thispagestyle{firstpage}
\vspace*{-0.9cm}
\begin{minipage}{0.9\textwidth}
    {\huge \color{blue} PolyModels Hub Ltd}
    \\
    \vspace{-0.15cm}

    71-75 Shelton Street
    \\
    London, WC2H 9JQ
    \\
    Company No. 14882310
    \\
    info@polymodelshub.com
    \\
\end{minipage}
\begin{minipage}{0.115\textwidth}
    \begin{figure}[H]
        \vspace{-0.5cm}\includegraphics[width=\textwidth]{PMHlogo.png}
    \end{figure}
\end{minipage}
\vspace{0.5cm}

\begin{minipage}{1\textwidth}
    \textbf{Report name}: \textcolor{blue}{ \large PMH Internal: Approach to solving PDE problems} \\
    \textbf{Report ID}: \textcolor{blue}{ CDCLINE\_01 }\\

\end{minipage}%

\vspace{0.0cm}

\noindent
\textbf{Last updated}: \today\\
\textbf{Authored by}: Samuel Andersson\\
\textbf{Reviewed by}: -\\

\vspace{0.5cm}


\rule{\textwidth}{0.1pt}


\begin{center} \textbf{{\large Report Summary}} \end{center}

In this report, we present approaches to solving types of PDE problems that are common in the pharmaceutical industry, comparing them in 3 principle aspects: ease of implementation, computational efficiency and accuracy.
First, we present a simple PDE problem that is representative of a common pharmaceutical process, the residence time distribution in a continuous powder mixer. We then move on to more complex problems, such as the fluid flow in a packed bed, and the diffusion of a solute through a liquid.
Then we compare computational implementations of these problems in a couple of software packages: py-pde, Scipy FDiff, PyClaw, and MoL.jl.


\rule{\textwidth}{0.1pt}

\vspace{1cm}

{\center{\textbf{\large Table of Contents}}
    \vspace*{-1.3cm}
    \renewcommand\contentsname{}
    \addcontentsline{}{}{}
    \addtocontents{toc}{~\hfill\textbf{Page}\par}

    \tableofcontents
}

\newpage{}

\section{Problem Sets}
\subsection{Feeder: Advection-Diffusion}
\subsubsection{V1: single pde}
\subsubsection{V2: coupled pde and ode}
\subsection{Fixed Bed Incompressible Fluid: Advection-Diffusion-Sorption}
\subsubsection{V1: coupled pdes}
\subsubsection{V2: coupled pde and ode}
\subsection{Fixed Bed Compressible Fluid: Advection-Diffusion-Sorption}

\section{Software Libraries}
\subsection{Py-PDE}
\subsection{Scipy FDiff}
\subsection{PyClaw}
\subsection{MoL.jl}

\section{Solving PDEs}
\subsection{Frameworks}
\subsubsection{Method of Lines}

\subsection{Methods}
\subsubsection{Finite Difference}
\subsubsection{Finite Volume}
\subsubsection{Finite Element}
\subsubsection{Orthogonal Collocation}
\subsubsection{Other}

\newpage{}
\appendix
\section{Hopper Balance Equation}
\label{sec:hopper_balance}
Starting from the PFR mass balance equation
\begin{equation}
    \frac{\partial f}{\partial t} = - v(t)\frac{\partial f}{\partial x} + D\frac{\partial^2 f}{\partial z^2}
\end{equation}
we replace the spatial dimension with a normalized coordinate $x = z/L(t)$ where $L$ is the material holdup height in the hopper. We call $g$ the function obtained from $f$ by replacing variable $z$ with $x$. We can then write the following equalities

\begin{equation}
    \label{eq:pfr}
    \begin{split}
        \frac{\partial f(z, t)}{\partial z}     & = \frac{\partial g(x, t)}{\partial z}                                                                                     \\
                                                & = \frac{\partial g(x, t)}{\partial x} \frac{\partial x}{\partial z}                                                       \\
                                                & = \frac{1}{L(t)}\frac{\partial g(x, t)}{\partial x}                                                                       \\ & \\
        \frac{\partial^2 f(z, t)}{\partial z^2} & = \frac{1}{L(t)}\frac{\partial^2 g(x, t)}{\partial x \partial z}                                                          \\
                                                & = \frac{1}{L(t)^2}\frac{\partial^2 g(x, t)}{\partial x^2}                                                                 \\ &\\
        \frac{\partial f(z, t)}{\partial t}     & = \frac{dg(x, t)}{dt}                                                                                                     \\
                                                & = \frac{\partial g(x, t)}{ \partial t} + \frac{\partial g(x, t)}{\partial x}\frac{\partial x}{\partial t}                 \\
                                                & = \frac{\partial g(x, t)}{ \partial t} + \frac{\partial g(x, t)}{\partial x} \bigg(-\frac{1}{L(t)}x\frac{dL(t)}{dt}\bigg)
    \end{split}
\end{equation}

The final equation with boundary immobilized is obtained by replacing the partial derivatives in Equation \ref{eq:pfr} with the $g$ dependent expression according to the above equalities.

\begin{equation}
    \frac{\partial g}{\partial t} = - \frac{1}{L(t)}\bigg(v(t) - x\frac{dL(t)}{dt}\bigg)\frac{\partial g}{\partial x} + \frac{D}{L(t)^2}\frac{\partial^2 g}{\partial x^2}; \quad x \in [0, 1]
\end{equation}

\newpage{}
\bibliographystyle{apalike}
\bibliography{references}



\end{document}
